\chapter*{Conclusion}
\addcontentsline{toc}{chapter}{Conclusion}

\lettrine{P}{armi} les différentes méthodes de résolution de conflits aériens en-route, il nous a semblé pertinent de dégager celles qui sont les plus adaptées à une coopération homme-machine. Le framework élaboré par le laboratoire MAIAA prend en compte de façon particulièrement réaliste les méthodes actuelles de gestion du trafic aérien. Cependant, si on souhaite évoluer vers une implémentation de tels outils dans une salle de contrôle, la robustesse des solutions proposées sera une question centrale. Elle déterminera leur acceptation par les contrôleurs qui sont par nature particulièrement sensibles aux changements pouvant affecter la sécurité des opérations.

Nous avons restreint l'étude de la robustesse à la notion de stabilité d'une solution. En effet, la fiabilité (et la tolérance à l'erreur ?) est déjà prise en compte par le framework du laboratoire MAIAA au travers d'une incertitude sur les vitesses des avions. Ainsi notre approche vient compléter des aspects déjà pris en compte. De nombreux scenarii de perturbations d'une solution sont possibles. Nous nous sommes limités à 2 cas bien définis : la panne radio et l'impossibilité pour un avion d'exécuter une manœuvre. Ces derniers nous ont semblé pertinents et sont parmi les plus simples à modéliser dans le framework.

Enfin nous avons produit un outil opérationnel et évolutif, pouvant être facilement intégré à la bibliothèque de code du laboratoire MAIAA, capable de conduire des test de robustesse en évaluant la réparabilité des solutions. Ce test est conduit à l'aide d'un algorithme de recuit simulé qui nous a été suggéré par le laboratoire. Cet algorithme s'est révélé particulièrement rapide dans la fourniture de nouvelles solutions. Nous avons cherché à l'optimiser en exploitant les abondantes statistiques produites par notre outil. Il en ressort que [bien que l'algorithme soit certainement encore optimisable, il fournit une information assez précise quant à la robustesse d'une solution]. Les test que nous avons conduit confirment donc la validité de cette approche dans l'évaluation de la robustesse.

Il reste malgré tout des cas de perturbations qui n'ont pas été étudiés ici, tel que l'interdiction d'une zone (militaire par exemple). Notre outil, par sa flexibilité, permettra une intégration future de nombreux autre cas de perturbations. Les test conduits sur les algorithmes CP et EA suggèrent que [premiers résultats sur la robustesse des solutions CP et EA]. 
Cependant l'algorithme de recuit simulé dispose de plusieurs paramètres qui pourront être affinés afin qu'il fournisse une évaluation définitive de la robustesse. Nous ne pouvons donc pas conclure quant à la qualité des solutions issues de ces algorithmes, ce qui n'était de toute façon pas l'objectif de notre étude, mais nous espérons que notre apport aidera le laboratoire MAIAA dans son travail constant visant à les optimiser.

\thispagestyle{plain}