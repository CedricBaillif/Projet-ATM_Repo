\chapter*{Introduction}
\addcontentsline{toc}{chapter}{Introduction}

\lettrine{L}{a résolution} de conflits aériens en-route fait encore largement appel à l'expertise humaine. Le contrôleur n'est aidé que par des outils rudimentaires permettant d'extrapoler la trajectoire des avions à partir des positions radar passées. Cependant les progrès réalisés dans le domaine des méthodes numériques permettent d'envisager une nouvelle classe d'outils de résolution de conflits. Le laboratoire MAIAA \footnote{Le laboratoire "Mathématique Appliquée, Informatique et Automatique pour l'Aérien" est une des 4 unités de recherche de l'École Nationale de l'Aviation Civile. Il est composé de 3 groupes de recherche : Optimisation, Probabilité et Statistiques, et Automatique, partageant un domaine d'application commun : la gestion du trafic aérien.} travaille sur plusieurs de ces méthodes, telles que la programmation par contraintes, les algorithmes évolutionnistes, ou la recherche tabou. Chacun de ces algorithmes est capable de fournir en un temps raisonnable une solution de résolution de conflit impliquant jusqu'à 20 avions.

Loin de pouvoir remplacer un opérateur humain, ces outils ont vocation à servir d'aide à la décision aux contrôleurs. Avant d'imaginer une implémentation de ces méthodes, il est légitime de soulever les questions suivantes : en cas de modification imprévue du scénario de trafic (erreur humaine, perte de contact, événement météo comme l'apparition de turbulence ou de cumulonimbus), ces solutions sont-elles toujours opérantes ? A défaut, sont-elles facilement réparables, c'est à dire redeviennent-elles efficientes en ne les modifiant que légèrement ?
Ces questions sont relatives à la notion de robustesse d'une solution. Celle-ci se présente sous de multiples aspects qu'il convient d'expliciter avant d'envisager de l'évaluer. Notre étude se propose d'établir une méthode d'évaluation de la robustesse fondée sur la notion de stabilité de la solution. 

Dans un premier temps nous présenterons l'aspect théorique de la gestion de conflit automatisée. Ce problème d'une grande complexité combinatoire a été abordé de façon très différente par plusieurs laboratoires dans le monde. En particulier, le laboratoire MAIAA a développé un cadre de travail (framework) adapté au point de vue du contrôleur que nous présenterons. Ce framework sert d'environnement d'étude pour plusieurs méthodes : les algorithmes évolutionnistes d'une part, la programmation par contraintes d'autre part dont il sera question.

Nous définirons ensuite le périmètre dans lequel se place notre étude de la robustesse. Une solution sera considérée comme robuste si, suite à une perturbation, il est possible d'en trouver une nouvelle suffisamment proche à l'aide d'un algorithme peu optimal mais rapide : le recuit simulé. Une telle solution sera appelée « super-solution ». Nous avons souhaité envisager des scénarii qui soient aussi opérationnels que possible. Nous nous concentrerons sur la recherche de « super-solutions » pour des perturbations précises : impossibilité pour l'avion d'effectuer une manœuvre (évitement de zone), ou manœuvre imposée pour un avion (ce qui inclut le cas de la panne radio).

Nous présentons enfin notre outil de test de la robustesse, développé en JAVA. Il permet de conduire des tests sur des solutions en conformité avec le framework du laboratoire MAIAA. Les résultats de ces tests nous permettront de vérifier la validité de notre cadre d'évaluation de la robustesse. Notre approche du développement de l'outil ne s'est pas limitée aux aspects scientifiques, mais a également inclus un effort important de structure et de concision du code, ainsi que de développement d'une documentation afin de faciliter son appropriation par le laboratoire MAIAA.  Ceci pour permettre une amélioration ultérieure dont certaines pistes seront présentées. 

Le travail réalisé permettra, nous l'espérons, le développement de solutions plus performantes de résolution de conflit par le laboratoire MAIAA.

\thispagestyle{plain}